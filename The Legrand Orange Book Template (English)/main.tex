%%%%%%%%%%%%%%%%%%%%%%%%%%%%%%%%%%%%%%%%%
% The Legrand Orange Book
% LaTeX Template
% Version 2.1.1 (14/2/16)
%
% This template has been downloaded from:
% http://www.LaTeXTemplates.com
%
% Original author:
% Mathias Legrand (legrand.mathias@gmail.com) with modifications by:
% Vel (vel@latextemplates.com)
%
% License:
% CC BY-NC-SA 3.0 (http://creativecommons.org/licenses/by-nc-sa/3.0/)
%
% Compiling this template:
% This template uses biber for its bibliography and makeindex for its index.
% When you first open the template, compile it from the command line with the 
% commands below to make sure your LaTeX distribution is configured correctly:
%
% 1) pdflatex main
% 2) makeindex main.idx -s StyleInd.ist
% 3) biber main
% 4) pdflatex main x 2
%
% After this, when you wish to update the bibliography/index use the appropriate
% command above and make sure to compile with pdflatex several times 
% afterwards to propagate your changes to the document.
%
% This template also uses a number of packages which may need to be
% updated to the newest versions for the template to compile. It is strongly
% recommended you update your LaTeX distribution if you have any
% compilation errors.
%
% Important note:
% Chapter heading images should have a 2:1 width:height ratio,
% e.g. 920px width and 460px height.
%
%%%%%%%%%%%%%%%%%%%%%%%%%%%%%%%%%%%%%%%%%

%----------------------------------------------------------------------------------------
%	PACKAGES AND OTHER DOCUMENT CONFIGURATIONS
%----------------------------------------------------------------------------------------

\documentclass[11pt,fleqn]{book} % Default font size and left-justified equations
\usepackage{ctex}
%----------------------------------------------------------------------------------------

\input{structure} % Insert the commands.tex file which contains the majority of the structure behind the template

\begin{document}

%----------------------------------------------------------------------------------------
%	TITLE PAGE
%----------------------------------------------------------------------------------------

\begingroup
\thispagestyle{empty}
\begin{tikzpicture}[remember picture,overlay]
\coordinate [below=12cm] (midpoint) at (current page.north);
\node at (current page.north west)
{\begin{tikzpicture}[remember picture,overlay]
\node[anchor=north west,inner sep=0pt] at (0,0) {\includegraphics[width=\paperwidth]{background}}; % Background image
\draw[anchor=north] (midpoint) node [fill=ocre!30!white,fill opacity=0.6,text opacity=1,inner sep=1cm]{\Huge\centering\bfseries\sffamily\parbox[c][][t]{\paperwidth}{\centering 天文学中的概率统计\\[15pt] % Book title
{\Large Probability statistics in Astronomy}\\[20pt] % Subtitle
{\huge 朱信成}}}; % Author name
\end{tikzpicture}};
\end{tikzpicture}
\vfill
\endgroup

%----------------------------------------------------------------------------------------
%	COPYRIGHT PAGE
%----------------------------------------------------------------------------------------

\newpage
~\vfill
\thispagestyle{empty}

\noindent Copyright \copyright\ 2013 John Smith\\ % Copyright notice

\noindent \textsc{Published by Publisher}\\ % Publisher

\noindent \textsc{book-website.com}\\ % URL

\noindent Licensed under the Creative Commons Attribution-NonCommercial 3.0 Unported License (the ``License''). You may not use this file except in compliance with the License. You may obtain a copy of the License at \url{http://creativecommons.org/licenses/by-nc/3.0}. Unless required by applicable law or agreed to in writing, software distributed under the License is distributed on an \textsc{``as is'' basis, without warranties or conditions of any kind}, either express or implied. See the License for the specific language governing permissions and limitations under the License.\\ % License information

\noindent \textit{First printing, March 2013} % Printing/edition date

%----------------------------------------------------------------------------------------
%	TABLE OF CONTENTS
%----------------------------------------------------------------------------------------

%\usechapterimagefalse % If you don't want to include a chapter image, use this to toggle images off - it can be enabled later with \usechapterimagetrue

\chapterimage{chapter_head_1.pdf} % Table of contents heading image

\pagestyle{empty} % No headers

\tableofcontents % Print the table of contents itself

\cleardoublepage % Forces the first chapter to start on an odd page so it's on the right

\pagestyle{fancy} % Print headers again

%----------------------------------------------------------------------------------------
%	PART
%----------------------------------------------------------------------------------------

\part{Part One}

%----------------------------------------------------------------------------------------
%	CHAPTER 1
%----------------------------------------------------------------------------------------

\chapterimage{chapter_head_2.pdf} % Chapter heading image

\chapter{概率论基础}

\section{观测科学的不确定性}\index{p1c1}

拟合参数的不确定性来源:测量过程的随机误差、系统误差和模型引入的模型误差。不确定性并不总是一成不变的:
\begin{itemize}
    \item 随着知识的积累而减少
    \item 微观不确定但宏观确定
\end{itemize}

%------------------------------------------------
\section{随机事件与样本空间}\index{p1c2}
对随机现象的观测称为随机试验,具有以下特征:
\begin{itemize}
    \item[\romannumeral1]在相同的条件下,试验可以重复地进行。
    \item[\romannumeral2]每次试验的结果不止一个,但事先知道有哪些结果$ E_1 $ ,$ E_2 $ ,$ E_3 $……
    \item[\romannumeral3]在每次试验前,一般不能确定哪个事件会发生。特例:必然事件$ U $,不可能事件$ \varnothing $
\end{itemize}
\begin{definition}[样本空间]
    不能进一步分割的随机事件是基本事件,随机事件的全体基本事件所组成的集合称为试验的样本空间$ \Omega $  
\end{definition}
随机现象的来源:1.随机误差 2.物理量的内禀随机性,具有某种统计分布,所测到的只是统计平均值

\section{事件之间的相互关系及运算}
随机事件的关系包括:
\begin{itemize}
    \item 包含:$ A $ 发生必然$ B $ 发生,则说$ A $ 包含于$ B $ ,记$ A\subset B $ 
    \item 相等:若$ A\subset B $ 且$ B\subset A $ 则说$ A=B $ 
    \item 互斥:若$ A $ 与$ B $ 不可能同时发生,即$ A\cap B=\varnothing $ 则$ A $ 与$ B $ 互斥
    \item 互逆:若在任何一次试验中,$ A $ 与$ B $ 有且只有一个事件发生,即$ A\cup B=\Omega,~A\cap B=\varnothing $ 则称$ A $ 与$ B $ 互逆,记为$ A=\bar{B} $ ,互逆必互斥,反之不然
\end{itemize}

随机事件的运算包括:
\begin{itemize}
    \item 和:事件$ A_1 $ ,$ A_2 $ ,$ A_3 $ ……中至少有一个事件发生的事件称为事件的和,记为$ \cup_{i=1}^{n}A_i   $  
    \item 积:事件$ A_1 $ ,$ A_2 $ ,$ A_3 $ ……同时发生,称为事件的积,记为$ \cap_{i=1}^{n}A_i   $
    \item 差:事件$ A $ 发生而事件$ B $ 不发生,称为$ A $ 与$ B $ 的差,记作$ A-B $ 
\end{itemize}
事件的运算规则满足一般集合的运算规则:
\begin{itemize}
    \item 吸收律:$ A\subset B\Rightarrow A\cup B=B,AB=A $ 
    \item 对偶律:$ \overline{\cup A_k}=\cap \bar{A_k};\overline{\cap A_k}=\cup \bar{A_k} $ 
\end{itemize}

\section{概率的一般定义与性质}
为了定量描述随机试验中,随机事件发生的可能性大小,引入概率$ P(A) $ 

\begin{theorem}[概率三公理]
\begin{align}
& 0\leqslant P(A)\leqslant 1\\ 
& P(\Omega)=1\\
& if~A_i \cap A_j=\varnothing,~P(\cup A_k)=\sum P(A_k)
\end{align}
\end{theorem}


概率的主要性质:
\begin{itemize}
    \item 不可能事件的概率为0:$ P(\varnothing)=0 $
    \item 加法公式:$$ P(A\cup B)=P(A)+P(B)-P(A\cap B) $$ 
    \item 单调性:在同一全集$ \Omega $ 下$ A\subset B\Rightarrow P(A)\leqslant P(B) $ 
\end{itemize} 
\begin{corollary}[“多退少补公式”]
   $$ P(\cup A_k)= \sum P(A_k)-\sum\limits_{1\leqslant i<j\leqslant n}P(A_i A_j)+\sum\limits_{1\leqslant i<j<k\leqslant n}P(A_i A_j A_k)+\cdots+(-1)^{n-1}P(A_1\cdots A_n) $$
\end{corollary}
\begin{exercise}
    将n个同样的盒子和n个同样的球分别标号1-n,把n个球分别投入n个盒子,每盒一个球,问:至少有一个球的编号与盒子的编号相同的概率
\end{exercise}
\begin{remark}
    $ \frac{1}{e} $ 
\end{remark}
\section{条件概率及与之有关的三个公式}
\begin{definition}[条件概率]
    设$ A $ 、$ B $  为同一随机试验的两个事件,$ P(A)>0 $ 定义$$ P(B|A)=P(AB)/P(A) $$ 为事件$ A $ 发生的条件下事件$ B $ 发生的概率
    
\end{definition}

\begin{exercise}[抓阄的公平性]
    $ n $ 个球($ a $ 个新球,其余为旧球),每次取一个,无放回地抽取$ k(k<n) $ 次,求每次抽到新球的概率;若第一次抽到新球,第二次取到的是新球的概率
\end{exercise}

\begin{remark}
    令$ A $ 为第一次抽到新球,$ B $ 为第二次抽到新球,有:
    $$ P(A)=\frac{a}{n};~~P(B)=P(A)\frac{a-1}{n-1}+(1-p(A))\frac{a}{n-1}=\frac{a}{n} $$ 
    以此类推,每次抽到新球的概率均为$ \frac{a}{n} $ 

    若第一次抽到新球,则在该情况下$ P(B|A)=\frac{a-1}{n-1} $ 
\end{remark}
注意,条件概率定义式中通常$ P(AB) $ 是未知的,也不需要求解,$ P(B|A) $ 可以在$ A $ 发生后的样本空间中求得

\begin{corollary}[乘法公式]
    $$ P(A_1\cdots A_n)=P(A_1)P(A_2|A_1)P(A_3|A_1A_2)\cdots P(A_n|A_1\cdots A_{n-1}) $$ 
\end{corollary}

\begin{corollary}[全概率公式]
    如果事件 $ B_1 $ ,$ \cdots $, $ B_n $ 
    满足互斥完备集条件,即两两互斥且事件集的并为全集,则:
    $$ \forall A,~~P(A)=\sum\limits_{k=1}^{n} P(B_k)P(A|B_k) $$ 
\end{corollary}

\begin{corollary}[贝叶斯公式]
    因为 $ P(AB)=P(BA) $ ,利用乘法公式可得:
    $$ P(B|A)=\frac{P(A|B)P(B)}{P(A)} $$    
\end{corollary}
贝叶斯公式将条件概率中的“正”问题 $ P(A|B) $ 转化为“逆”问题 $ P(B|A) $ :

如果把 $ A $ 理解为观测数据, $ B $ 理解为某种理论模型, $ P(B) $ 是根据以往经验模型正确的概率(先验概率),那么 $ P(A|B) $ 即为在模型 $ B $ 正确的情况下取得观测数据 $ A $ 的概率(通过测量得到 $ A $ 通过模型 $ B $ (通常为分布)计算 $ P(A|B) $ )。 $ P(A) $ 可利用多次测量观察 $ A $ 的分布得到。运用贝叶斯公式,可以得到 $ P(B|A) $ 即在随机变量 $ A $ 取某值(或分布)时模型成立的概率(后验概率),是结果对模型的修正:
$$ P(B)\triangleq P(B|A_{previous}) $$
即前一次测量的后验概率是后一次测量的先验概率。
\\贝叶斯公式更完整的表述是:

设 $ B_1,B_2,\cdots,B_n $为互斥完备集,则 $ \forall $ 事件 $ A $ ,当 $ P(A)>0 $ 时,有
$$ P(B_i|A)=\frac{P(B_i)P(A|B_i)}{\sum\limits_{k=1}^{n}P(B_k)P(A|B_k) },~~i=1,2,\cdots,n. $$  

\section{相互独立的事件}
\begin{definition}[独立]
    称 $ A $ , $ B $ 为相互独立的事件,如果 $ P(AB)=P(A)\cdot P(B) $ 
\end{definition}

\section{Citation}\index{Citation}

This statement requires citation \cite{book_key}; this one is more specific \cite[122]{article_key}.

%------------------------------------------------

\section{Lists}\index{Lists}

Lists are useful to present information in a concise and/or ordered way\footnote{Footnote example...}.

\subsection{Numbered List}\index{Lists!Numbered List}

\begin{enumerate}
\item The first item
\item The second item
\item The third item
\end{enumerate}

\subsection{Bullet Points}\index{Lists!Bullet Points}

\begin{itemize}
\item The first item
\item The second item
\item The third item
\end{itemize}

\subsection{Descriptions and Definitions}\index{Lists!Descriptions and Definitions}

\begin{description}
\item[Name] Description
\item[Word] Definition
\item[Comment] Elaboration
\end{description}

%----------------------------------------------------------------------------------------
%	CHAPTER 2
%----------------------------------------------------------------------------------------

\chapter{In-text Elements}

\section{Theorems}\index{Theorems}

This is an example of theorems.

\subsection{Several equations}\index{Theorems!Several Equations}
This is a theorem consisting of several equations.

\begin{theorem}[Name of the theorem]
In $E=\mathbb{R}^n$ all norms are equivalent. It has the properties:
\begin{align}
& \big| ||\mathbf{x}|| - ||\mathbf{y}|| \big|\leq || \mathbf{x}- \mathbf{y}||\\
&  ||\sum_{i=1}^n\mathbf{x}_i||\leq \sum_{i=1}^n||\mathbf{x}_i||\quad\text{where $n$ is a finite integer}
\end{align}
\end{theorem}

\subsection{Single Line}\index{Theorems!Single Line}
This is a theorem consisting of just one line.

\begin{theorem}
A set $\mathcal{D}(G)$ in dense in $L^2(G)$, $|\cdot|_0$. 
\end{theorem}

%------------------------------------------------

\section{Definitions}\index{Definitions}

This is an example of a definition. A definition could be mathematical or it could define a concept.

\begin{definition}[Definition name]
Given a vector space $E$, a norm on $E$ is an application, denoted $||\cdot||$, $E$ in $\mathbb{R}^+=[0,+\infty[$ such that:
\begin{align}
& ||\mathbf{x}||=0\ \Rightarrow\ \mathbf{x}=\mathbf{0}\\
& ||\lambda \mathbf{x}||=|\lambda|\cdot ||\mathbf{x}||\\
& ||\mathbf{x}+\mathbf{y}||\leq ||\mathbf{x}||+||\mathbf{y}||
\end{align}
\end{definition}

%------------------------------------------------

\section{Notations}\index{Notations}

\begin{notation}
Given an open subset $G$ of $\mathbb{R}^n$, the set of functions $\varphi$ are:
\begin{enumerate}
\item Bounded support $G$;
\item Infinitely differentiable;
\end{enumerate}
a vector space is denoted by $\mathcal{D}(G)$. 
\end{notation}

%------------------------------------------------

\section{Remarks}\index{Remarks}

This is an example of a remark.

\begin{remark}
The concepts presented here are now in conventional employment in mathematics. Vector spaces are taken over the field $\mathbb{K}=\mathbb{R}$, however, established properties are easily extended to $\mathbb{K}=\mathbb{C}$.
\end{remark}

%------------------------------------------------

\section{Corollaries}\index{Corollaries}

This is an example of a corollary.

\begin{corollary}[Corollary name]
The concepts presented here are now in conventional employment in mathematics. Vector spaces are taken over the field $\mathbb{K}=\mathbb{R}$, however, established properties are easily extended to $\mathbb{K}=\mathbb{C}$.
\end{corollary}

%------------------------------------------------

\section{Propositions}\index{Propositions}

This is an example of propositions.

\subsection{Several equations}\index{Propositions!Several Equations}

\begin{proposition}[Proposition name]
It has the properties:
\begin{align}
& \big| ||\mathbf{x}|| - ||\mathbf{y}|| \big|\leq || \mathbf{x}- \mathbf{y}||\\
&  ||\sum_{i=1}^n\mathbf{x}_i||\leq \sum_{i=1}^n||\mathbf{x}_i||\quad\text{where $n$ is a finite integer}
\end{align}
\end{proposition}

\subsection{Single Line}\index{Propositions!Single Line}

\begin{proposition} 
Let $f,g\in L^2(G)$; if $\forall \varphi\in\mathcal{D}(G)$, $(f,\varphi)_0=(g,\varphi)_0$ then $f = g$. 
\end{proposition}

%------------------------------------------------

\section{Examples}\index{Examples}

This is an example of examples.

\subsection{Equation and Text}\index{Examples!Equation and Text}

\begin{example}
Let $G=\{x\in\mathbb{R}^2:|x|<3\}$ and denoted by: $x^0=(1,1)$; consider the function:
\begin{equation}
f(x)=\left\{\begin{aligned} & \mathrm{e}^{|x|} & & \text{si $|x-x^0|\leq 1/2$}\\
& 0 & & \text{si $|x-x^0|> 1/2$}\end{aligned}\right.
\end{equation}
The function $f$ has bounded support, we can take $A=\{x\in\mathbb{R}^2:|x-x^0|\leq 1/2+\epsilon\}$ for all $\epsilon\in\intoo{0}{5/2-\sqrt{2}}$.
\end{example}

\subsection{Paragraph of Text}\index{Examples!Paragraph of Text}

\begin{example}[Example name]
\lipsum[2]
\end{example}

%------------------------------------------------

\section{Exercises}\index{Exercises}

This is an example of an exercise.

\begin{exercise}
This is a good place to ask a question to test learning progress or further cement ideas into students' minds.
\end{exercise}

%------------------------------------------------

\section{Problems}\index{Problems}

\begin{problem}
What is the average airspeed velocity of an unladen swallow?
\end{problem}

%------------------------------------------------

\section{Vocabulary}\index{Vocabulary}

Define a word to improve a students' vocabulary.

\begin{vocabulary}[Word]
Definition of word.
\end{vocabulary}

%----------------------------------------------------------------------------------------
%	PART
%----------------------------------------------------------------------------------------

\part{Part Two}

%----------------------------------------------------------------------------------------
%	CHAPTER 3
%----------------------------------------------------------------------------------------

\chapterimage{chapter_head_1.pdf} % Chapter heading image

\chapter{Presenting Information}

\section{Table}\index{Table}

\begin{table}[h]
\centering
\begin{tabular}{l l l}
\toprule
\textbf{Treatments} & \textbf{Response 1} & \textbf{Response 2}\\
\midrule
Treatment 1 & 0.0003262 & 0.562 \\
Treatment 2 & 0.0015681 & 0.910 \\
Treatment 3 & 0.0009271 & 0.296 \\
\bottomrule
\end{tabular}
\caption{Table caption}
\end{table}

%------------------------------------------------

\section{Figure}\index{Figure}

\begin{figure}[h]
\centering\includegraphics[scale=0.5]{placeholder}
\caption{Figure caption}
\end{figure}

%----------------------------------------------------------------------------------------
%	BIBLIOGRAPHY
%----------------------------------------------------------------------------------------

\chapter*{Bibliography}
\addcontentsline{toc}{chapter}{\textcolor{ocre}{Bibliography}}
\section*{Books}
\addcontentsline{toc}{section}{Books}
\printbibliography[heading=bibempty,type=book]
\section*{Articles}
\addcontentsline{toc}{section}{Articles}
\printbibliography[heading=bibempty,type=article]

%----------------------------------------------------------------------------------------
%	INDEX
%----------------------------------------------------------------------------------------

\cleardoublepage
\phantomsection
\setlength{\columnsep}{0.75cm}
\addcontentsline{toc}{chapter}{\textcolor{ocre}{Index}}
\printindex

%----------------------------------------------------------------------------------------

\end{document}
